% group4.tex
%tes AB
\section{Splinter Group 4: Probe combination / Flow-down}

This provides at the moment a short summary of discussions so far, effectively
reflecting what is also in the google doc presentation.

\subsubsection*{Coordinators}
\begin{itemize}
 \item Alain Blanchard, Martin Kunz
\end{itemize}
\subsubsection*{Participants}
\begin{itemize}
 \item \ldots [pls add yourself]
\end{itemize}

\subsubsection*{Aim}

\begin{itemize}
\item Define the data-flow and theoretical concepts/recommendations for probe combination
\item Define the validation tests for each “atom” in the atomistic code
\item Make recommendations for splinter-1 on code 
\end{itemize}



\subsection{Workflows}

\mku{add workflow pictures}

\subsection{Probe combination formalism}

\mku{discussion here based on http://arxiv.org/abs/1601.05779}

We would much prefer if WL and GC used a compatible approach to simplify probe combination!

\subsection{Probe combination activities inside Euclid}

The following list is atm in random order, based on answers received after
a relatively wide `spam query'.

\begin{itemize}
\item {\bf Martin Kilbinger}: experience with CFHTLenS, code is cosmo\_pmc (http://cosmopmc.info), partially reused in cosmoSIS, is also in WL IST TG and so has limited time (but is interested)
\item {\bf Fil Abdalla \& Michael McLeod}: working on probe combination (including calibration of photo-z and galaxy Cl), Michael may be able to join TG4.
\item {\bf Melita Carbone}: link with GC forecasting in IST (sg3)
\item {\bf Karim Benabed}: contact for weak lensing forecasting in IST (sg2)
\item {\bf Mariana Penna-Lima}: co-author of NumCosmo
\item {\bf Fabien Lacasa}: in the process of joining Euclid, has worked on combining cluster number counts and galaxy clustering, cf http://arxiv.org/abs/1603.00918
\end{itemize}

In addition, several members of other survey active in probe combination are also in Euclid (see following section).
We should use their expertise!

\subsection{Probe combination activities in other surveys}

Should we propose an inter-mission collaboration on probe combination (both intra-mission and inter-mission)?

\subsubsection{SKA}

\begin{itemize}
\item Contacts: Roy Maartens and Mario Santos
\item References: chapters by Takahashi et al, Kitching et al and Bacon et al in SKA cosmology science chapters
\end{itemize}

\subsubsection{SPHEREx \& WFIRST}

\begin{itemize}
\item Contacts: Olivier Dor\'e  [Euclid member]
\item Planning to use Cosmolike developed by Krause and Eifler, may be interested in working with us.
\end{itemize}



\subsubsection{LSST}

\begin{itemize}
\item Contacts: Rachel Bean [Euclid member], Pierre Astier
\item Effort led by Joe Zuntz and Elisabeth Krause
\item Expect to use cosmolike and cosmoSIS in a larger framework with tested cosmology tools
\item References: \mku{what was that doc link again?}
\end{itemize}



\subsubsection{DES}

\begin{itemize}
\item Contacts: No ‘official’ request yet, but infos from Jochen Weller, Fabien Lacasa, Martin Kilbinger
\item Other DES members also in Euclid: Isaac Tutusaus, group in Barcelona (Pablo Fosalba, Martin Crocce, Francisco Castander)
\item As far as we know they will be using cosmolike and cosmoSIS
\end{itemize}



\subsubsection{DESI}

\begin{itemize}
\item Contacts: No ‘official’ request yet, Martin Kilbinger, Will Percival, Risa Wechsler?
\end{itemize}



